
\documentclass{article}

\usepackage{amsmath}
\usepackage{amssymb}
\usepackage{geometry}
\usepackage{fancyhdr}
\usepackage{graphicx}
\usepackage{array}
\usepackage{multirow}
\usepackage{tikz}
\usepackage{booktabs}
\usepackage{float}



% for custom subsection
\usepackage{titlesec}
% package for enumerate with letters
\usepackage{enumitem}

\titleformat{\subsection}
  {\normalfont\fontfamily{phv}\fontsize{14}{17}}{\thesubsection}{1em}{}

\geometry{margin=1in}
\pagestyle{fancy}
\fancyhf{}
\rhead{Kyle Wodehouse}
\lhead{phys228}
\chead{lab 3}
\title{\bfseries lab 3}
\author{Kyle Wodehouse}
\rfoot{\thepage}
% 1 -> 5 from least to most descriptive
\setcounter{tocdepth}{1}

\begin{document}
\maketitle



\section*{c and d}

\begin{align*}
    \delta A &= A \sqrt{ \left( \frac{\delta l}{l} \right)^2 + \left( \frac{\delta h}{h} \right)^2 } \\
    &= 296.74 \sqrt{ \left( \frac{0.01}{19.60} \right)^2 + \left( \frac{0.01}{15.14} \right)^2 } \\
    &= 0.247662
\end{align*}

\vspace{2em}

We know for this setup

\[ C = \frac{K \epsilon_0 A}{D} \]

Therefore the graph of $C$ vs $1/D$ should be linear with a slope of $K  \epsilon_0  A$.

\vspace{2em}

\begin{align*}
    K \epsilon_0 A &= m = 6.328 \times 10^{-13} \\
    K &= \frac{(6.328 \times 10^{-13}) \ \textnormal{F} \times \textnormal{m}}{8.85 \times 10^{-12} \times \left( 296.74 \textnormal{cm}^2 \times \frac{1 \textnormal{m}^2}{10000 \textnormal{cm}^2} \right)} \\
    &= 2.416 
\end{align*}

% now calculating error in K

\begin{align*}
    \delta K &= K \sqrt{ \left( \frac{\delta A}{A} \right)^2 + \left( \frac{\delta m}{m} \right)^2 } \\
    &= 2.416 \sqrt{ \left( \frac{0.248}{296.74} \right)^2 + \left( \frac{2.42 \times 10^{-14}}{6.35 \times 10^{-13}} \right)^2 } \\
    &= 0.092
\end{align*}

\begin{table}[H]
    \centering
    \begin{tabular}{ccc}
    & 100\% Area & 50.7\% of Area \\
    \toprule
    Capacitence (nF) \hspace{1em} \vline& 0.78       & 0.50           \\ 
    \end{tabular}
    \caption{Comparison of Whole Area and 50.7\% of Area}
    \label{tab:area_comparison}
\end{table}

% capacitence calculations

\begin{align*}
    C_{series} &= \frac{1}{\frac{1}{C_1} + \frac{1}{C_2}} \ \textnormal{nF} \\
    &= \frac{1}{\frac{1}{3.52 \textnormal{nF}} + \frac{1}{3.52 \textnormal{nF}}} \\
    &= 1.76 \ \textnormal{nF} \\
\end{align*}

\begin{align*}
    C_{parallel} &= C_1 + C_2 + C_3 \ \textnormal{nF} \\
    &= 3.52 \textnormal{nF} + 3.52 \textnormal{nF} + 3.52 \textnormal{nF} \\
    &= 10.56 \ \textnormal{nF} \\
\end{align*}



\end{document}